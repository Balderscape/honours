\documentclass[12pt, a4paper, oneside]{report} 


\usepackage{amsmath} % AMS Math Package
\usepackage{amsthm} % Theorem Formatting
\usepackage{amssymb}	% Math symbols such as \mathbb
\usepackage{graphicx} % Allows for eps images
\usepackage{setspace}
\usepackage{hyperref}
\usepackage[utf8]{inputenc}
\usepackage[
margin=1.25 in,
includefoot,
footskip=30pt,
]{geometry}
\usepackage{chemfig}
\usepackage[sort&compress,numbers,super]{natbib}
\usepackage[version=3]{mhchem}
\usepackage[none]{hyphenat}
\usepackage{honours}

% This defines the "body" environment
\newenvironment{preamble}
{
	\clearpage
	\pagenumbering{roman}
	\setcounter{page}{1}
}

\newenvironment{body}
{
	\clearpage
	\pagenumbering{arabic}
	\setcounter{page}{1}
}


\spacing{1.5} 
\bibliographystyle{achemso}

% This defines the "abbreviations" environment
\newcommand{\abbrlabel}[1]{\makebox[3cm][l]{\textbf{#1}\ \dotfill}}
\newenvironment{abbreviations}{\begin{list}{}{\renewcommand{\makelabel}{\abbrlabel}}}{\end{list}}

\author{Paul D. Solomon}
\title{Protein Folding in Higher Dimensions}
\subtitle{Using higher dimensional space to improve speed of protein folding simulations}
\university{The University of Sydney}
\degree{Bachelor of Science (Advanced) (Honours) }
\school{School of Chemistry}
\submit{November 2017}


\begin{document}

\maketitle

% The "preamble" environment used roman numbering
\begin{preamble}
	
	\chapter*{Abstract}
	
	The first sentence of your abstract is critical, It is usually a brief 
	informational statement of your the results reported in your thesis. In 
	principle, the reader should have to go no further if the subject is of little 
	interest to him or her. Your abstract does not include a restatement of your 
	title. It should contain specific data. For example, a general statement that 
	an important measurement was made is insufficient; include the actual results 
	and their uncertainties.
	\\\\
	For a thesis you may like to write two or three paragraphs in your abstract. 
	That is, you may wish to use a paragraph for each of the major results in your 
	thesis.
	\\\\
	
	\newpage
	
	\chapter*{Acknowledgements and Statement of Contribution}
	You must acknowledge all assistance from staff and others on a separate 
	page of the thesis, immediately after the title page and before any general 
	acknowledgements, explaining their role briefly but precisely. You must give 
	details of contributors to the project. For example, were you part of a team 
	who performed the experiment, did someone write parts of the computer code, 
	take the observations, build some of the apparatus, or do some of the analytic 
	work? If any of your Honours research built on results previously obtained in 
	another research project (for example a TSP project) these results must be 
	clearly highlighted. Here you must also acknowledge any assistance that was 
	provided in thesis preparation. You must provide a signed and dated statement 
	of originality of the form: 
	\\\\
	\centerline{I certify that this report contains work carried out by myself}
	\centerline{except where otherwise acknowledge}
	\\\\\
	Signed:\\
	\rule[1em]{25em}{0.5pt} % This prints a line for the signature
	
	Date:\\
	\rule[1em]{25em}{0.5pt} % This prints a line to write the date
	\newpage
	
	\section*{Acknowledgements}
	This is the general statement thanking people. It is the appropriate place to 
	thank your supervisor, your family, your research group, your office mates, 
	the rest of the honours cohort, the pizza place down the road...
	\\\\
	
	\newpage
	
	% Here you make your table of contents
	\tableofcontents
	\newpage
	
	% Here you list your figures (or tables)
	\listoffigures
	\newpage
	% you may also like to list abbreviations used as a list of abbreviations or 
	%as a glossary
	\chapter*{Abbreviations}
	\begin{center}
		
		\begin{abbreviations}
			\item[ESR] Electron Spin Resonance
			\item[NMR] Nuclear Magnetic Resonance
		\end{abbreviations}
		
	\end{center}
\end{preamble}

% Now begin the body of your thesis 
% The "body" environment used arabic numbering
% Here each chapter is a separate document and latex incorporates them all 
% when it compiles the document
%
\begin{body}
	\chapter{Introduction}
Your introductory chapter. This should provide a framework for the rest of 
your thesis. It should introduce the context of your research and its 
importance. You should discuss previous work in the field and appropriate 
methodology for your project. Ideally your introduction should also spell out:
\begin{enumerate}
\item what you want to do
\item why it is important
\item how you are going to do it
\end{enumerate}

\section{Specific Aims}
You may like to use sub-headings for background, methodology and aims. You may 
also like to provide a paragraph at the end of the introduction outlining the 
structure of your thesis.

 \newpage


	\chapter{Theoretical or Experimental Methods}\label{Chapter:Two}
% the label lets you corss-reference your chapters
\section{Introduction}
For a physical or theoretical chemistry thesis Chapter 2 typically outlines 
the theoretical or experimental methods you have used, derived or constructed. 
This format is different for organic theses where the experimental methods 
section is usually at the end. The first paragraph of the chapter should 
orient your reader (marker) and let them know what to expect.

\section{The Path Integral Formulation of Quantum     
Mechanics}\label{Chapter:theoryPIformalism}
% again the label is so that you can cross-reference.
The path integral formalism, while mathematically equivalent to the
Schr\"{o}dinger formalism is conceptually very
different.\cite{PIMCBerne1982,Tuckerman2010,Feynman2012} Within the
Schr\"{o}dinger picture a quantum mechanical system is described by a
wavefunction, $\Psi$.  This wavefunction propagates in time, $t$, according to 
the time-dependent Schr\"{o}dinger equation, a wave equation:
form:
\begin{equation}\label{theory:1:propagator}
\left | \Psi(t) \right> = \exp\left( -\frac{i\hat{H}t}{\hbar}\right) \left | 
\Psi(0) \right> = \hat{U}(t) \left | \Psi(0) \right>
\end{equation}
where $\hat{H}$ is the Hamiltonian, $\hbar$ is the reduced Planck's constant,
$\Psi(0)$ and $\Psi(t)$ are the wavefunction at time $t=0$ and time $t=t$
respectively and the operator $\hat{U}$ is known as the propagator.  Equation 
\ref{theory:1:propagator} propagates $\Psi$, 
initially at a position $\mathbf{R}$ at $t=0$, as a wave. As this happens the 
system delocalises, until it is observed again at $t=t$, at which point the 
$\Psi$ localises at a new position. The probability of finding $\Psi$ at this 
new position is proportional to $\Psi^2$.
\\\\
Feynman's path integral interpretation of quantum mechanics more closely 
resembles the classical picture.  The position of a classical system is always 
well defined: given a system at an initial position $\mathbf{R}$, it will 
propagate according to the principle of least action, moving along the path 
which minimises the quantity known as the 
\textit{action}.\cite{Goldstein2013} Within the
path integral formalism, rather than propagating along the path of least
action, the system propagates along all possible paths simultaneously.
The probability of a given path being taken is given by a probability 
amplitude (see below).  The total probability of the system propagating from an
initial position to a final position is given by the sum of this amplitude
over each possible path connecting these two positions. It is this summation
over paths that gives the path integral method its name.  In the following 
section the mathematical formalism behind path integrals will be presented.

\subsection{The Canonical Density Matrix}
And so on...
\\\\
You may like to include a paragraph leading into the next chapter, Chapter 
\ref{Chapter:Three}.

\newpage

	\chapter{Comparison of Path Integral Methods}\label{Chapter:Three}
\section{Introduction}
This chapter compares the different path integral formalisms derived in 
Chapter \ref{Chapter:Two}. Here you can think of it as a combined ``Results 
and Discussion'' chapter. You may prefer separate chapters. Again, you should 
orient your reader in terms of what to expect from the chapter.
\\\\
\section{Path Integral Method 1}
You should present your results logically. You can divide this Chapter up any 
it makes sense to you (and ideally to make the  most sense to your reader). 
You should include sufficient detail to reproduce your results.
\\\\
\section{Path Integral Method 2}
You should use figures and tables to help reinforce your main points. For 
example, Figure  shows the refinement of Chin action 
path integral parameters for water. These have been includes as pdfs but you 
can use other formats.
\\

\section{Conclusions}
You may like to include a separate conclusions for this Chapter, or for each 
subsection of the chapter.

\newpage



%	\input{Chapter4.tex}
%	
\end{body}

\begin{preamble}
	
	% Now for bibliography and appendices (if applicable). Again these are using 
	%roman numerals for the page numbers. 
	\begin{spacing}{1.0}
		
		\bibliography{honours}
		
	\end{spacing}
	%\input{Appendix.tex}
\end{preamble}
	
\end{document}