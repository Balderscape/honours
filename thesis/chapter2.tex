\chapter{Theoretical or Experimental Methods}\label{Chapter:Two}
% the label lets you corss-reference your chapters
\section{Introduction}
For a physical or theoretical chemistry thesis Chapter 2 typically outlines 
the theoretical or experimental methods you have used, derived or constructed. 
This format is different for organic theses where the experimental methods 
section is usually at the end. The first paragraph of the chapter should 
orient your reader (marker) and let them know what to expect.

\section{The Path Integral Formulation of Quantum     
Mechanics}\label{Chapter:theoryPIformalism}
% again the label is so that you can cross-reference.
The path integral formalism, while mathematically equivalent to the
Schr\"{o}dinger formalism is conceptually very
different.\cite{PIMCBerne1982,Tuckerman2010,Feynman2012} Within the
Schr\"{o}dinger picture a quantum mechanical system is described by a
wavefunction, $\Psi$.  This wavefunction propagates in time, $t$, according to 
the time-dependent Schr\"{o}dinger equation, a wave equation:
form:
\begin{equation}\label{theory:1:propagator}
\left | \Psi(t) \right> = \exp\left( -\frac{i\hat{H}t}{\hbar}\right) \left | 
\Psi(0) \right> = \hat{U}(t) \left | \Psi(0) \right>
\end{equation}
where $\hat{H}$ is the Hamiltonian, $\hbar$ is the reduced Planck's constant,
$\Psi(0)$ and $\Psi(t)$ are the wavefunction at time $t=0$ and time $t=t$
respectively and the operator $\hat{U}$ is known as the propagator.  Equation 
\ref{theory:1:propagator} propagates $\Psi$, 
initially at a position $\mathbf{R}$ at $t=0$, as a wave. As this happens the 
system delocalises, until it is observed again at $t=t$, at which point the 
$\Psi$ localises at a new position. The probability of finding $\Psi$ at this 
new position is proportional to $\Psi^2$.
\\\\
Feynman's path integral interpretation of quantum mechanics more closely 
resembles the classical picture.  The position of a classical system is always 
well defined: given a system at an initial position $\mathbf{R}$, it will 
propagate according to the principle of least action, moving along the path 
which minimises the quantity known as the 
\textit{action}.\cite{Goldstein2013} Within the
path integral formalism, rather than propagating along the path of least
action, the system propagates along all possible paths simultaneously.
The probability of a given path being taken is given by a probability 
amplitude (see below).  The total probability of the system propagating from an
initial position to a final position is given by the sum of this amplitude
over each possible path connecting these two positions. It is this summation
over paths that gives the path integral method its name.  In the following 
section the mathematical formalism behind path integrals will be presented.

\subsection{The Canonical Density Matrix}
And so on...
\\\\
You may like to include a paragraph leading into the next chapter, Chapter 
\ref{Chapter:Three}.

\newpage
